% Options for packages loaded elsewhere
\PassOptionsToPackage{unicode}{hyperref}
\PassOptionsToPackage{hyphens}{url}
%
\documentclass[
]{article}
\usepackage{lmodern}
\usepackage{amssymb,amsmath}
\usepackage{ifxetex,ifluatex}
\ifnum 0\ifxetex 1\fi\ifluatex 1\fi=0 % if pdftex
  \usepackage[T1]{fontenc}
  \usepackage[utf8]{inputenc}
  \usepackage{textcomp} % provide euro and other symbols
\else % if luatex or xetex
  \usepackage{unicode-math}
  \defaultfontfeatures{Scale=MatchLowercase}
  \defaultfontfeatures[\rmfamily]{Ligatures=TeX,Scale=1}
\fi
% Use upquote if available, for straight quotes in verbatim environments
\IfFileExists{upquote.sty}{\usepackage{upquote}}{}
\IfFileExists{microtype.sty}{% use microtype if available
  \usepackage[]{microtype}
  \UseMicrotypeSet[protrusion]{basicmath} % disable protrusion for tt fonts
}{}
\makeatletter
\@ifundefined{KOMAClassName}{% if non-KOMA class
  \IfFileExists{parskip.sty}{%
    \usepackage{parskip}
  }{% else
    \setlength{\parindent}{0pt}
    \setlength{\parskip}{6pt plus 2pt minus 1pt}}
}{% if KOMA class
  \KOMAoptions{parskip=half}}
\makeatother
\usepackage{xcolor}
\IfFileExists{xurl.sty}{\usepackage{xurl}}{} % add URL line breaks if available
\IfFileExists{bookmark.sty}{\usepackage{bookmark}}{\usepackage{hyperref}}
\hypersetup{
  pdftitle={On the Prediction of Deliquencies},
  hidelinks,
  pdfcreator={LaTeX via pandoc}}
\urlstyle{same} % disable monospaced font for URLs
\usepackage[margin=1in]{geometry}
\usepackage{color}
\usepackage{fancyvrb}
\newcommand{\VerbBar}{|}
\newcommand{\VERB}{\Verb[commandchars=\\\{\}]}
\DefineVerbatimEnvironment{Highlighting}{Verbatim}{commandchars=\\\{\}}
% Add ',fontsize=\small' for more characters per line
\usepackage{framed}
\definecolor{shadecolor}{RGB}{248,248,248}
\newenvironment{Shaded}{\begin{snugshade}}{\end{snugshade}}
\newcommand{\AlertTok}[1]{\textcolor[rgb]{0.94,0.16,0.16}{#1}}
\newcommand{\AnnotationTok}[1]{\textcolor[rgb]{0.56,0.35,0.01}{\textbf{\textit{#1}}}}
\newcommand{\AttributeTok}[1]{\textcolor[rgb]{0.77,0.63,0.00}{#1}}
\newcommand{\BaseNTok}[1]{\textcolor[rgb]{0.00,0.00,0.81}{#1}}
\newcommand{\BuiltInTok}[1]{#1}
\newcommand{\CharTok}[1]{\textcolor[rgb]{0.31,0.60,0.02}{#1}}
\newcommand{\CommentTok}[1]{\textcolor[rgb]{0.56,0.35,0.01}{\textit{#1}}}
\newcommand{\CommentVarTok}[1]{\textcolor[rgb]{0.56,0.35,0.01}{\textbf{\textit{#1}}}}
\newcommand{\ConstantTok}[1]{\textcolor[rgb]{0.00,0.00,0.00}{#1}}
\newcommand{\ControlFlowTok}[1]{\textcolor[rgb]{0.13,0.29,0.53}{\textbf{#1}}}
\newcommand{\DataTypeTok}[1]{\textcolor[rgb]{0.13,0.29,0.53}{#1}}
\newcommand{\DecValTok}[1]{\textcolor[rgb]{0.00,0.00,0.81}{#1}}
\newcommand{\DocumentationTok}[1]{\textcolor[rgb]{0.56,0.35,0.01}{\textbf{\textit{#1}}}}
\newcommand{\ErrorTok}[1]{\textcolor[rgb]{0.64,0.00,0.00}{\textbf{#1}}}
\newcommand{\ExtensionTok}[1]{#1}
\newcommand{\FloatTok}[1]{\textcolor[rgb]{0.00,0.00,0.81}{#1}}
\newcommand{\FunctionTok}[1]{\textcolor[rgb]{0.00,0.00,0.00}{#1}}
\newcommand{\ImportTok}[1]{#1}
\newcommand{\InformationTok}[1]{\textcolor[rgb]{0.56,0.35,0.01}{\textbf{\textit{#1}}}}
\newcommand{\KeywordTok}[1]{\textcolor[rgb]{0.13,0.29,0.53}{\textbf{#1}}}
\newcommand{\NormalTok}[1]{#1}
\newcommand{\OperatorTok}[1]{\textcolor[rgb]{0.81,0.36,0.00}{\textbf{#1}}}
\newcommand{\OtherTok}[1]{\textcolor[rgb]{0.56,0.35,0.01}{#1}}
\newcommand{\PreprocessorTok}[1]{\textcolor[rgb]{0.56,0.35,0.01}{\textit{#1}}}
\newcommand{\RegionMarkerTok}[1]{#1}
\newcommand{\SpecialCharTok}[1]{\textcolor[rgb]{0.00,0.00,0.00}{#1}}
\newcommand{\SpecialStringTok}[1]{\textcolor[rgb]{0.31,0.60,0.02}{#1}}
\newcommand{\StringTok}[1]{\textcolor[rgb]{0.31,0.60,0.02}{#1}}
\newcommand{\VariableTok}[1]{\textcolor[rgb]{0.00,0.00,0.00}{#1}}
\newcommand{\VerbatimStringTok}[1]{\textcolor[rgb]{0.31,0.60,0.02}{#1}}
\newcommand{\WarningTok}[1]{\textcolor[rgb]{0.56,0.35,0.01}{\textbf{\textit{#1}}}}
\usepackage{graphicx,grffile}
\makeatletter
\def\maxwidth{\ifdim\Gin@nat@width>\linewidth\linewidth\else\Gin@nat@width\fi}
\def\maxheight{\ifdim\Gin@nat@height>\textheight\textheight\else\Gin@nat@height\fi}
\makeatother
% Scale images if necessary, so that they will not overflow the page
% margins by default, and it is still possible to overwrite the defaults
% using explicit options in \includegraphics[width, height, ...]{}
\setkeys{Gin}{width=\maxwidth,height=\maxheight,keepaspectratio}
% Set default figure placement to htbp
\makeatletter
\def\fps@figure{htbp}
\makeatother
\setlength{\emergencystretch}{3em} % prevent overfull lines
\providecommand{\tightlist}{%
  \setlength{\itemsep}{0pt}\setlength{\parskip}{0pt}}
\setcounter{secnumdepth}{-\maxdimen} % remove section numbering

\title{On the Prediction of Deliquencies}
\author{}
\date{\vspace{-2.5em}}

\begin{document}
\maketitle

\hypertarget{introduction}{%
\section{Introduction}\label{introduction}}

This is an R Markdown document. Markdown is a simple formatting syntax
for authoring HTML, PDF, and MS Word documents. For more details on
using R Markdown see \url{http://rmarkdown.rstudio.com}.

When you click the \textbf{Knit} button a document will be generated
that includes both content as well as the output of any embedded R code
chunks within the document. You can embed an R code chunk like this:

\hypertarget{libraries}{%
\section{Libraries}\label{libraries}}

\begin{Shaded}
\begin{Highlighting}[]
\KeywordTok{library}\NormalTok{(smbinning)}
\KeywordTok{library}\NormalTok{(InformationValue)}
\KeywordTok{library}\NormalTok{(randomForest)}
\NormalTok{set.seed =}\StringTok{ }\DecValTok{1}
\end{Highlighting}
\end{Shaded}

\hypertarget{read-data}{%
\section{Read Data}\label{read-data}}

\begin{Shaded}
\begin{Highlighting}[]
\NormalTok{data <-}\StringTok{ }\KeywordTok{read.csv}\NormalTok{(}\DataTypeTok{file =} \StringTok{"mart_delinquency.csv"}\NormalTok{,}
                 \DataTypeTok{colClasses=}\KeywordTok{c}\NormalTok{(}
                              \StringTok{"ORDER_DT_HOUR"}\NormalTok{              =}\StringTok{"factor"}\NormalTok{,}
                              \StringTok{"ORDER_DT_MONTH"}\NormalTok{             =}\StringTok{"factor"}\NormalTok{,}
                              \StringTok{"PROFILEDESC"}\NormalTok{                =}\StringTok{"factor"}\NormalTok{,}
                              \StringTok{"TIME_BETWEEN_REG_FIRST_BIN"}\NormalTok{ =}\StringTok{"factor"}\NormalTok{,}
                              \StringTok{"DEM_BIN"}\NormalTok{                    =}\StringTok{"factor"}\NormalTok{,}
                              \StringTok{"DIS_BIN"}\NormalTok{                    =}\StringTok{"factor"}\NormalTok{,}
                              \StringTok{"SHIPTYPE"}\NormalTok{                   =}\StringTok{"factor"}\NormalTok{,}
                              \StringTok{"STATE"}\NormalTok{                      =}\StringTok{"factor"}\NormalTok{,}
                              \StringTok{"STATE_BIN"}\NormalTok{                  =}\StringTok{"factor"}\NormalTok{,}
                              \StringTok{"STATE_BIN_2"}\NormalTok{                =}\StringTok{"factor"}\NormalTok{,}
                              \StringTok{"ORDER_DT_HOUR_BIN"}\NormalTok{          =}\StringTok{"factor"}\NormalTok{,}
                              \StringTok{"USERGROUP"}\NormalTok{                  =}\StringTok{"factor"}
\NormalTok{                             )}
\NormalTok{                )}
\end{Highlighting}
\end{Shaded}

\hypertarget{creating-training-and-test-sets}{%
\section{Creating Training and Test
sets}\label{creating-training-and-test-sets}}

\begin{Shaded}
\begin{Highlighting}[]
\CommentTok{# Create Training Data}
\NormalTok{input_ones <-}\StringTok{ }\NormalTok{data[}\KeywordTok{which}\NormalTok{(data}\OperatorTok{$}\NormalTok{OUTCOME }\OperatorTok{==}\StringTok{ }\DecValTok{1}\NormalTok{), ]  }\CommentTok{# all 1's}
\NormalTok{input_zeros <-data[}\KeywordTok{which}\NormalTok{(data}\OperatorTok{$}\NormalTok{OUTCOME }\OperatorTok{==}\StringTok{ }\DecValTok{0}\NormalTok{), ]  }\CommentTok{# all 0's}
\CommentTok{# Create Test Data}
\NormalTok{input_ones_train_rows  <-}\StringTok{ }\KeywordTok{sample}\NormalTok{(}\DecValTok{1}\OperatorTok{:}\KeywordTok{nrow}\NormalTok{(input_ones), }\FloatTok{0.7} \OperatorTok{*}\StringTok{ }\KeywordTok{nrow}\NormalTok{(input_ones))  }\CommentTok{# 1's for training}
\NormalTok{input_zeros_train_rows <-}\StringTok{ }\KeywordTok{sample}\NormalTok{(}\DecValTok{1}\OperatorTok{:}\KeywordTok{nrow}\NormalTok{(input_zeros), }\FloatTok{0.7} \OperatorTok{*}\StringTok{ }\KeywordTok{nrow}\NormalTok{(input_zeros))  }\CommentTok{# 0's for training. Pick as many 0's as 1's}
\CommentTok{# }
\NormalTok{training_ones <-}\StringTok{ }\NormalTok{input_ones[input_ones_train_rows, ]  }
\NormalTok{training_zeros <-}\StringTok{ }\NormalTok{input_zeros[input_zeros_train_rows, ]}
\NormalTok{data_train <-}\StringTok{ }\KeywordTok{rbind}\NormalTok{(training_ones, training_zeros)  }\CommentTok{# row bind the 1's and 0's }
\CommentTok{# }
\NormalTok{test_ones <-}\StringTok{ }\NormalTok{input_ones[}\OperatorTok{-}\NormalTok{input_ones_train_rows, ]}
\NormalTok{test_zeros <-}\StringTok{ }\NormalTok{input_zeros[}\OperatorTok{-}\NormalTok{input_zeros_train_rows, ]}
\NormalTok{data_test <-}\StringTok{ }\KeywordTok{rbind}\NormalTok{(test_ones, test_zeros)  }\CommentTok{# row bind the 1's and 0's }
\end{Highlighting}
\end{Shaded}

\pagebreak

\hypertarget{calculating-the-information-value-iv-of-independent-variables}{%
\section{Calculating the Information Value (IV) of independent
variables}\label{calculating-the-information-value-iv-of-independent-variables}}

In this section we calculate the IV of each independent variable. The IV
is a measure of the relationship between an independent and the
dependent variable.As a rule of thumb, independent variables with an IV
less than 0.02 are ignored. First we select the names of variables to be
investigated.

\begin{Shaded}
\begin{Highlighting}[]
\NormalTok{vars <-}\StringTok{ }\KeywordTok{c}\NormalTok{(}\StringTok{"PROFILEDESC"}\NormalTok{, }
          \StringTok{"TIME_BETWEEN_REG_FIRST_BIN"}\NormalTok{, }
          \StringTok{"DEM_BIN"}\NormalTok{, }
          \StringTok{"DIS_BIN"}\NormalTok{, }
          \StringTok{"SHIPTYPE"}\NormalTok{, }
          \StringTok{"STATE_BIN_2"}\NormalTok{, }
          \StringTok{"ORDER_DT_HOUR_BIN"}\NormalTok{,}
          \StringTok{"STATE"}\NormalTok{,}
          \StringTok{"USERGROUP"}\NormalTok{,}
          
          \StringTok{"ORDER_DT_MONTH_VAL"}\NormalTok{, }
          \StringTok{"DEPT_WOA"}\NormalTok{, }
          \StringTok{"SUBCLASS_VAL"}\NormalTok{,}
          \StringTok{"USERGROUP_WOA"}\NormalTok{,}
          \StringTok{"STATE_WOA"}
\NormalTok{)}
\end{Highlighting}
\end{Shaded}

And we calculate the corresponding IVs using the smbinning package.

\begin{Shaded}
\begin{Highlighting}[]
\NormalTok{vars <-}\StringTok{ }\KeywordTok{c}\NormalTok{(vars)}
\NormalTok{df.iv <-}\StringTok{ }\KeywordTok{data.frame}\NormalTok{(}\DataTypeTok{vars=}\NormalTok{vars, }\DataTypeTok{IV=}\KeywordTok{numeric}\NormalTok{(}\KeywordTok{length}\NormalTok{(vars)))}
\ControlFlowTok{for}\NormalTok{(var }\ControlFlowTok{in}\NormalTok{ vars)\{}
  \CommentTok{#smb <- smbinning.factor(data_train, y="OUTCOME", x=var, maxcat = 60)  # WOE table}
  \CommentTok{#if(class(smb) != "character")\{ # heck if some error occured}
  \CommentTok{#  df.iv[df.iv$vars == var, "IV"] <- smb$iv}
  \CommentTok{#\}}
\NormalTok{\}}
\CommentTok{#df.iv <- df.iv[order(-df.iv$IV),]}
\CommentTok{#df.iv}
\end{Highlighting}
\end{Shaded}

\pagebreak

\hypertarget{order-attributes}{%
\subsection{Order Attributes}\label{order-attributes}}

\hypertarget{product-type}{%
\subsubsection{Product Type}\label{product-type}}

A particular order may have multiple products. Products can be
categorized to (i) Departments (Dept), (ii) Classes (Class), and (iii)
Subclasses (Subclass).We review the IV of each of these attributes. We
first look at the Product Department. The column DEPT\_WOA contains the
Weight of Evidence (WoE) for each department. The column DEPT\_VAL
contains the average deliquency for each department.

\begin{Shaded}
\begin{Highlighting}[]
\NormalTok{var =}\StringTok{ 'DEPT_WOA'}
\NormalTok{smb <-}\StringTok{ }\KeywordTok{smbinning}\NormalTok{(data_train, }\DataTypeTok{y=}\StringTok{"OUTCOME"}\NormalTok{, }\DataTypeTok{x=}\NormalTok{var)}
\NormalTok{smb}\OperatorTok{$}\NormalTok{iv}
\end{Highlighting}
\end{Shaded}

\begin{verbatim}
## [1] 0.041
\end{verbatim}

\begin{Shaded}
\begin{Highlighting}[]
\NormalTok{var =}\StringTok{ 'DEPT_VAL'}
\NormalTok{smb <-}\StringTok{ }\KeywordTok{smbinning}\NormalTok{(data_train, }\DataTypeTok{y=}\StringTok{"OUTCOME"}\NormalTok{, }\DataTypeTok{x=}\NormalTok{var)}
\NormalTok{smb}\OperatorTok{$}\NormalTok{iv}
\end{Highlighting}
\end{Shaded}

\begin{verbatim}
## [1] 0.0411
\end{verbatim}

We repeat the same exercise for class.

\begin{Shaded}
\begin{Highlighting}[]
\NormalTok{var =}\StringTok{ 'CLASS_WOA'}
\NormalTok{smb <-}\StringTok{ }\KeywordTok{smbinning}\NormalTok{(data_train, }\DataTypeTok{y=}\StringTok{"OUTCOME"}\NormalTok{, }\DataTypeTok{x=}\NormalTok{var)}
\NormalTok{smb}\OperatorTok{$}\NormalTok{iv}
\end{Highlighting}
\end{Shaded}

\begin{verbatim}
## [1] 0.0252
\end{verbatim}

\begin{Shaded}
\begin{Highlighting}[]
\NormalTok{var =}\StringTok{ 'CLASS_VAL'}
\NormalTok{smb <-}\StringTok{ }\KeywordTok{smbinning}\NormalTok{(data_train, }\DataTypeTok{y=}\StringTok{"OUTCOME"}\NormalTok{, }\DataTypeTok{x=}\NormalTok{var)}
\NormalTok{smb}\OperatorTok{$}\NormalTok{iv}
\end{Highlighting}
\end{Shaded}

\begin{verbatim}
## [1] 0.0252
\end{verbatim}

We repeat the same exercise for Subclass.

\begin{Shaded}
\begin{Highlighting}[]
\NormalTok{var =}\StringTok{ 'SUBCLASS_WOA'}
\NormalTok{smb <-}\StringTok{ }\KeywordTok{smbinning}\NormalTok{(data_train, }\DataTypeTok{y=}\StringTok{"OUTCOME"}\NormalTok{, }\DataTypeTok{x=}\NormalTok{var)}
\NormalTok{smb}\OperatorTok{$}\NormalTok{iv}
\end{Highlighting}
\end{Shaded}

\begin{verbatim}
## [1] 0.006
\end{verbatim}

\begin{Shaded}
\begin{Highlighting}[]
\NormalTok{var =}\StringTok{ 'SUBCLASS_VAL'}
\NormalTok{smb <-}\StringTok{ }\KeywordTok{smbinning}\NormalTok{(data_train, }\DataTypeTok{y=}\StringTok{"OUTCOME"}\NormalTok{, }\DataTypeTok{x=}\NormalTok{var)}
\NormalTok{smb}\OperatorTok{$}\NormalTok{iv}
\end{Highlighting}
\end{Shaded}

\begin{verbatim}
## [1] 0.006
\end{verbatim}

\hypertarget{order-value}{%
\subsubsection{Order Value}\label{order-value}}

We investigate the Order Value and perform supervised binning. The
central idea is to find those cutpoints that maximize the difference
between the groups. Using `smbinning' package we can quickly find the
optimal cutpoints in seconds and evaluate the relationship with the
target variable using metrics such as Weight of Evidence and Information
Value.

\begin{Shaded}
\begin{Highlighting}[]
\NormalTok{var <-}\StringTok{ 'DEM'}
\NormalTok{smb <-}\StringTok{ }\KeywordTok{smbinning}\NormalTok{(data_train, }\DataTypeTok{y=}\StringTok{"OUTCOME"}\NormalTok{, }\DataTypeTok{x=}\NormalTok{var, }\DataTypeTok{p =} \FloatTok{0.25}\NormalTok{)}
\NormalTok{smb}\OperatorTok{$}\NormalTok{ctree}
\end{Highlighting}
\end{Shaded}

\begin{verbatim}
## 
## Model formula:
## OUTCOME ~ DEM
## 
## Fitted party:
## [1] root
## |   [2] DEM <= 486.56: 0.142 (n = 27309, err = 3335.2)
## |   [3] DEM > 486.56: 0.168 (n = 81892, err = 11428.0)
## 
## Number of inner nodes:    1
## Number of terminal nodes: 2
\end{verbatim}

\begin{Shaded}
\begin{Highlighting}[]
\CommentTok{#smb$ivtable}

\CommentTok{#smbinning.plot(smb, option="dist")}

\CommentTok{#smbinning.plot(smb, option="goodrate")}
\end{Highlighting}
\end{Shaded}

We created three bins based on the following cutpoints: (i) \$174, and
(ii) \$324.

\begin{Shaded}
\begin{Highlighting}[]
\KeywordTok{aggregate}\NormalTok{(data_train}\OperatorTok{$}\NormalTok{OUTCOME, }\DataTypeTok{by=}\KeywordTok{list}\NormalTok{(data_train}\OperatorTok{$}\NormalTok{DEM_BIN), }\DataTypeTok{FUN=}\NormalTok{mean)}
\end{Highlighting}
\end{Shaded}

\begin{verbatim}
##   Group.1         x
## 1   Bin 1 0.1038214
## 2   Bin 2 0.1282027
## 3   Bin 3 0.1668184
\end{verbatim}

\begin{Shaded}
\begin{Highlighting}[]
\NormalTok{var =}\StringTok{ 'DEM'}
\NormalTok{smb <-}\StringTok{ }\KeywordTok{smbinning}\NormalTok{(data_train, }\DataTypeTok{y=}\StringTok{"OUTCOME"}\NormalTok{, }\DataTypeTok{x=}\NormalTok{var)}
\NormalTok{smb}\OperatorTok{$}\NormalTok{iv}
\end{Highlighting}
\end{Shaded}

\begin{verbatim}
## [1] 0.0132
\end{verbatim}

\begin{Shaded}
\begin{Highlighting}[]
\NormalTok{var =}\StringTok{ 'DEM_BIN'}
\NormalTok{smb <-}\StringTok{ }\KeywordTok{smbinning.factor}\NormalTok{(data_train, }\DataTypeTok{y=}\StringTok{"OUTCOME"}\NormalTok{, }\DataTypeTok{x=}\NormalTok{var)}
\NormalTok{smb}\OperatorTok{$}\NormalTok{iv}
\end{Highlighting}
\end{Shaded}

\begin{verbatim}
## [1] 0.0136
\end{verbatim}

\hypertarget{discount}{%
\subsubsection{Discount}\label{discount}}

The variable DIS represent the \% of discount applied to the order. We
perform a form of supervised binning as follows:

\begin{Shaded}
\begin{Highlighting}[]
\NormalTok{var =}\StringTok{ 'DIS'}
\NormalTok{smb <-}\StringTok{ }\KeywordTok{smbinning}\NormalTok{(data_train, }\DataTypeTok{y=}\StringTok{"OUTCOME"}\NormalTok{, }\DataTypeTok{x=}\NormalTok{var, }\DataTypeTok{p =} \FloatTok{0.20}\NormalTok{)}
\NormalTok{smb}\OperatorTok{$}\NormalTok{ctree}
\end{Highlighting}
\end{Shaded}

\begin{verbatim}
## 
## Model formula:
## OUTCOME ~ DIS
## 
## Fitted party:
## [1] root
## |   [2] DIS <= 0.0961: 0.177 (n = 63250, err = 9216.8)
## |   [3] DIS > 0.0961
## |   |   [4] DIS <= 0.1737: 0.150 (n = 23598, err = 3014.6)
## |   |   [5] DIS > 0.1737: 0.128 (n = 22353, err = 2502.3)
## 
## Number of inner nodes:    2
## Number of terminal nodes: 3
\end{verbatim}

\begin{Shaded}
\begin{Highlighting}[]
\NormalTok{var =}\StringTok{ 'DIS'}
\NormalTok{smb <-}\StringTok{ }\KeywordTok{smbinning}\NormalTok{(data_train, }\DataTypeTok{y=}\StringTok{"OUTCOME"}\NormalTok{, }\DataTypeTok{x=}\NormalTok{var)}
\NormalTok{smb}\OperatorTok{$}\NormalTok{iv}
\end{Highlighting}
\end{Shaded}

\begin{verbatim}
## [1] 0.0237
\end{verbatim}

\begin{Shaded}
\begin{Highlighting}[]
\NormalTok{var =}\StringTok{ 'DIS_BIN'}
\NormalTok{smb <-}\StringTok{ }\KeywordTok{smbinning.factor}\NormalTok{(data_train, }\DataTypeTok{y=}\StringTok{"OUTCOME"}\NormalTok{, }\DataTypeTok{x=}\NormalTok{var)}
\NormalTok{smb}\OperatorTok{$}\NormalTok{iv}
\end{Highlighting}
\end{Shaded}

\begin{verbatim}
## [1] 0.0235
\end{verbatim}

\begin{Shaded}
\begin{Highlighting}[]
\KeywordTok{aggregate}\NormalTok{(data_train}\OperatorTok{$}\NormalTok{OUTCOME, }\DataTypeTok{by=}\KeywordTok{list}\NormalTok{(data_train}\OperatorTok{$}\NormalTok{DIS_BIN), }\DataTypeTok{FUN=}\NormalTok{mean)}
\end{Highlighting}
\end{Shaded}

\begin{verbatim}
##   Group.1         x
## 1   Bin 1 0.1770751
## 2   Bin 2 0.1495981
## 3   Bin 3 0.1231556
\end{verbatim}

\hypertarget{order-ship-type}{%
\subsubsection{Order Ship Type}\label{order-ship-type}}

\begin{Shaded}
\begin{Highlighting}[]
\NormalTok{var =}\StringTok{ 'SHIPTYPE'}
\NormalTok{smb <-}\StringTok{ }\KeywordTok{smbinning.factor}\NormalTok{(data_train, }\DataTypeTok{y=}\StringTok{"OUTCOME"}\NormalTok{, }\DataTypeTok{x=}\NormalTok{var)}
\NormalTok{smb}\OperatorTok{$}\NormalTok{iv}
\end{Highlighting}
\end{Shaded}

\begin{verbatim}
## [1] 0.0099
\end{verbatim}

\hypertarget{time}{%
\subsubsection{Time}\label{time}}

\begin{Shaded}
\begin{Highlighting}[]
\NormalTok{var =}\StringTok{ 'ORDER_DT_HOUR'}
\NormalTok{smb <-}\StringTok{ }\KeywordTok{smbinning.factor}\NormalTok{(data_train, }\DataTypeTok{y=}\StringTok{"OUTCOME"}\NormalTok{, }\DataTypeTok{x=}\NormalTok{var, }\DataTypeTok{maxcat =} \DecValTok{50}\NormalTok{)}
\NormalTok{smb}\OperatorTok{$}\NormalTok{iv}
\end{Highlighting}
\end{Shaded}

\begin{verbatim}
## [1] 0.004
\end{verbatim}

\begin{Shaded}
\begin{Highlighting}[]
\NormalTok{var =}\StringTok{ 'ORDER_DT_MONTH'}
\NormalTok{smb <-}\StringTok{ }\KeywordTok{smbinning.factor}\NormalTok{(data_train, }\DataTypeTok{y=}\StringTok{"OUTCOME"}\NormalTok{, }\DataTypeTok{x=}\NormalTok{var, }\DataTypeTok{maxcat =} \DecValTok{50}\NormalTok{)}
\NormalTok{smb}\OperatorTok{$}\NormalTok{iv}
\end{Highlighting}
\end{Shaded}

\begin{verbatim}
## [1] 0.013
\end{verbatim}

\pagebreak

\hypertarget{buyer-attributes}{%
\subsection{Buyer Attributes}\label{buyer-attributes}}

\hypertarget{time-between-registration-and-first-purchase}{%
\subsubsection{Time between Registration and First
Purchase}\label{time-between-registration-and-first-purchase}}

\begin{Shaded}
\begin{Highlighting}[]
\NormalTok{var =}\StringTok{ 'TIME_BETWEEN_REG_FIRST'}
\NormalTok{smb <-}\StringTok{ }\KeywordTok{smbinning}\NormalTok{(data_train, }\DataTypeTok{y=}\StringTok{"OUTCOME"}\NormalTok{, }\DataTypeTok{x=}\NormalTok{var)}
\NormalTok{smb}\OperatorTok{$}\NormalTok{iv}
\end{Highlighting}
\end{Shaded}

\begin{verbatim}
## [1] 0.0871
\end{verbatim}

\begin{Shaded}
\begin{Highlighting}[]
\NormalTok{var =}\StringTok{ 'TIME_BETWEEN_REG_FIRST_BIN'}
\NormalTok{smb <-}\StringTok{ }\KeywordTok{smbinning.factor}\NormalTok{(data_train, }\DataTypeTok{y=}\StringTok{"OUTCOME"}\NormalTok{, }\DataTypeTok{x=}\NormalTok{var)}
\NormalTok{smb}\OperatorTok{$}\NormalTok{iv}
\end{Highlighting}
\end{Shaded}

\begin{verbatim}
## [1] 0.0653
\end{verbatim}

\hypertarget{geography}{%
\subsubsection{Geography}\label{geography}}

\begin{Shaded}
\begin{Highlighting}[]
\NormalTok{var =}\StringTok{ 'STATE'}
\NormalTok{smb <-}\StringTok{ }\KeywordTok{smbinning.factor}\NormalTok{(data_train, }\DataTypeTok{y=}\StringTok{"OUTCOME"}\NormalTok{, }\DataTypeTok{x=}\NormalTok{var, }\DataTypeTok{maxcat =} \DecValTok{100}\NormalTok{)}
\NormalTok{smb}\OperatorTok{$}\NormalTok{iv}
\end{Highlighting}
\end{Shaded}

\begin{verbatim}
## [1] 0.0205
\end{verbatim}

\begin{Shaded}
\begin{Highlighting}[]
\NormalTok{var =}\StringTok{ 'STATE_BIN'}
\NormalTok{smb <-}\StringTok{ }\KeywordTok{smbinning.factor}\NormalTok{(data_train, }\DataTypeTok{y=}\StringTok{"OUTCOME"}\NormalTok{, }\DataTypeTok{x=}\NormalTok{var, }\DataTypeTok{maxcat =} \DecValTok{100}\NormalTok{)}
\NormalTok{smb}\OperatorTok{$}\NormalTok{iv}
\end{Highlighting}
\end{Shaded}

\begin{verbatim}
## [1] 0.0194
\end{verbatim}

\begin{Shaded}
\begin{Highlighting}[]
\NormalTok{var =}\StringTok{ 'STATE_BIN_2'}
\NormalTok{smb <-}\StringTok{ }\KeywordTok{smbinning.factor}\NormalTok{(data_train, }\DataTypeTok{y=}\StringTok{"OUTCOME"}\NormalTok{, }\DataTypeTok{x=}\NormalTok{var, }\DataTypeTok{maxcat =} \DecValTok{100}\NormalTok{)}
\NormalTok{smb}\OperatorTok{$}\NormalTok{iv}
\end{Highlighting}
\end{Shaded}

\begin{verbatim}
## [1] 0.0154
\end{verbatim}

\hypertarget{usergroup}{%
\subsubsection{UserGroup}\label{usergroup}}

\begin{Shaded}
\begin{Highlighting}[]
\NormalTok{var =}\StringTok{ 'USERGROUP'}
\NormalTok{smb <-}\StringTok{ }\KeywordTok{smbinning.factor}\NormalTok{(data_train, }\DataTypeTok{y=}\StringTok{"OUTCOME"}\NormalTok{, }\DataTypeTok{x=}\NormalTok{var, }\DataTypeTok{maxcat =} \DecValTok{100}\NormalTok{)}
\NormalTok{smb}\OperatorTok{$}\NormalTok{iv}
\end{Highlighting}
\end{Shaded}

\begin{verbatim}
## [1] 0.0396
\end{verbatim}

\begin{Shaded}
\begin{Highlighting}[]
\NormalTok{var =}\StringTok{ 'USERGROUP_WOA'}
\NormalTok{smb <-}\StringTok{ }\KeywordTok{smbinning}\NormalTok{(data_train, }\DataTypeTok{y=}\StringTok{"OUTCOME"}\NormalTok{, }\DataTypeTok{x=}\NormalTok{var)}
\NormalTok{smb}\OperatorTok{$}\NormalTok{iv}
\end{Highlighting}
\end{Shaded}

\begin{verbatim}
## [1] 0.0254
\end{verbatim}

\hypertarget{fitting-a-logistic-regression-model}{%
\section{Fitting a Logistic Regression
Model}\label{fitting-a-logistic-regression-model}}

\begin{Shaded}
\begin{Highlighting}[]
\NormalTok{logitMod <-}\StringTok{ }\KeywordTok{glm}\NormalTok{(OUTCOME }\OperatorTok{~}\StringTok{ }\NormalTok{ORDER_DT_MONTH }\OperatorTok{+}\StringTok{ }
\StringTok{                          }\NormalTok{DEPT_WOA }\OperatorTok{+}
\StringTok{                          }\NormalTok{CLASS_WOA }\OperatorTok{+}
\StringTok{                          }\NormalTok{DEM }\OperatorTok{+}\StringTok{ }
\StringTok{                          }\NormalTok{DIS }\OperatorTok{+}
\StringTok{                          }\NormalTok{TIME_BETWEEN_REG_FIRST }\OperatorTok{+}
\StringTok{                          }\NormalTok{PROFILEDESC }\OperatorTok{+}
\StringTok{                          }\NormalTok{STATE }\OperatorTok{+}
\StringTok{                          }\NormalTok{USERGROUP, }
                \DataTypeTok{data=}\NormalTok{data_train, }
                \DataTypeTok{family=}\KeywordTok{binomial}\NormalTok{(}\DataTypeTok{link=}\StringTok{"logit"}\NormalTok{)}
\NormalTok{                )}
\CommentTok{#summary(logitMod)}
\NormalTok{predicted <-}\StringTok{ }\KeywordTok{predict}\NormalTok{(logitMod, data_test, }\DataTypeTok{type=}\StringTok{"response"}\NormalTok{)  }\CommentTok{# predicted scores}
\CommentTok{#}
\NormalTok{i <-}\StringTok{ }\OperatorTok{!}\KeywordTok{is.na}\NormalTok{(predicted) }\OperatorTok{&}\StringTok{ }\NormalTok{predicted }\OperatorTok{>=}\StringTok{ }\FloatTok{0.30}
\KeywordTok{sum}\NormalTok{(i)}
\end{Highlighting}
\end{Shaded}

\begin{verbatim}
## [1] 1727
\end{verbatim}

\begin{Shaded}
\begin{Highlighting}[]
\KeywordTok{sum}\NormalTok{(i)}\OperatorTok{/}\KeywordTok{length}\NormalTok{(i)}
\end{Highlighting}
\end{Shaded}

\begin{verbatim}
## [1] 0.03690013
\end{verbatim}

\begin{Shaded}
\begin{Highlighting}[]
\KeywordTok{plotROC}\NormalTok{(data_test}\OperatorTok{$}\NormalTok{OUTCOME, predicted)}
\end{Highlighting}
\end{Shaded}

\includegraphics{main_files/figure-latex/IV12233-1.pdf}

\begin{Shaded}
\begin{Highlighting}[]
\KeywordTok{plot}\NormalTok{(}\KeywordTok{sort}\NormalTok{(predicted))}
\end{Highlighting}
\end{Shaded}

\includegraphics{main_files/figure-latex/IV1233-1.pdf}

\end{document}
